\include{includes/common_start}
\tutnr{2}

\section{DEA $\rightarrow$ Regex}
\subsection{Konstruktion eines regulären Ausdrucks}
\begin{frame}
\frametitle{Konstruktion eines RA aus einem DEA}
Wir wissen: Zu jedem DEA gibt es einen regulären Ausdruck, der genau die Sprache beschreibt, die der Automat akzeptiert. Wie konstruiert man nun diesen RA aus dem DEA?\\[0.6cm]
\textbf{Idee:} Betrachte die Sprachen $L_{q_r,i,q_t}$, definiert als \( w \in \Sigma^*\) mit $w$ überführt $q_r$ in $q_t$ unter Benutzung der Zwischenzustände $\{q_1,\ldots,q_i\}$
\begin{itemize}
\item Es ist $L = \cup_{f\in F} L_{s,n,f}$
\item Es ist weiterhin $L_{q_r,i+1,q_t} = L_{q_r,i,q_t} \cup (L_{q_r,i,q_{i+1}}(L_{q_{i+1},i,q_{i+1}})^*L_{q_{i+1},i,q_t})$
\item Letztlich ist $L_{q_r, 0, q_t}$ immer regulär, denn das sind die Zeichen, mit denen man von $q_r$ nach $q_t$ kommt, ohne weitere Zustände zu verwenden (sowie $\varepsilon$, falls $r = t$).
\item Unter Benutzung dieser Punkte kann man nun zu einem DEA einen regulären Ausdruck konstruieren.
\end{itemize}
\end{frame}

% Lösungen nicht vorher anzeigen
\setbeamercovered{invisible}

\begin{frame}
\frametitle{Beispiele zum Verständnis}
\begin{figure}[H]
\begin{center}
\begin{tikzpicture}[scale=0.6,node distance=1.9cm,shorten >=1pt,auto]
\node[state,initial]   (q_1)                {$q_1$};
\node[state,accepting] (q_2) [right of=q_0] {$q_2$};
\path[->]	(q_1) 	edge 			node {$0$} 		(q_2)
			edge [loop above]	node {$1$}		()
		(q_2)	edge [loop above]	node {$0$,$1$}		();
\end{tikzpicture}
\end{center}
\end{figure}

Was sind hier jeweils:

\begin{itemize}
\item $L_{q_1,0,q_1}$\pause $ = (1\cup\varepsilon)$
\item $L_{q_1,0,q_2}$\pause $ = (0)$
\item $L_{q_1,1,q_1}$\pause $ = (1^*)$
\item $L_{q_2,1,q_2}$\pause $ = (0\cup1\cup\varepsilon)$
\item $L_{q_2,2,q_2}$\pause $ = (0\cup1)^*$
\item $L_{q_1,2,q_2}$\pause $ = 1^*0(0\cup1)^*$
\end{itemize}
\end{frame}

\setbeamercovered{transparent}

\begin{frame}
\frametitle{Ausführliche Konstruktion}
\begin{figure}[H]
\begin{center}
\begin{tikzpicture}[scale=0.6,node distance=1.9cm,shorten >=1pt,auto]
\node[state,initial]   (q_1)                {$q_1$};
\node[state,accepting] (q_2) [right of=q_0] {$q_2$};
\path[->]	(q_1) 	edge 			node {$0$} 		(q_2)
			edge [loop above]	node {$1$}		()
		(q_2)	edge [loop above]	node {$0$,$1$}		();
\end{tikzpicture}
\end{center}
\end{figure}
\begin{enumerate}
\item $L_{q_1,2,q_2} = L_{q_1,1,q_2}\cup(L_{q_1,1,q_2}(L_{q_2,1,q_2})^*L_{q_2,1,q_2})$
\item $L_{q_1,1,q_2} = L_{q_1,0,q_2}\cup(L_{q_1,0,q_1}(L_{q_1,0,q_1})^*L_{q_1,0,q_2}) = 0\cup((1\cup\varepsilon)(1\cup\varepsilon)^*0)$
\item $L_{q_2,1,q_2} = L_{q_2,0,q_2}\cup(L_{q_2,0,q_1}(L_{q_1,0,q_1})^*L_{q_1,0,q_2})$ $ = (0\cup1\cup\varepsilon)$
\item Also: $L_{q_1,2,q_2} = 0\cup(1\cup\varepsilon)(1\cup\varepsilon)^*0\cup((0\cup((1\cup\varepsilon)(1\cup\varepsilon)^*0))(0\cup1\cup\varepsilon)^*(0\cup1\cup\varepsilon))$
\item Vereinfacht: $1^*0(0\cup1)^*$
\end{enumerate}
\end{frame}

\begin{frame}
\frametitle{Aufgabe}
Bestimmen Sie nach der Methode aus der Vorlesung (Satz 2.14 im Skript zur Vorlesung) einen
regulären Ausdruck für die von folgendem Automaten erkannte Sprache. Geben Sie dabei alle benötigten Zwischenergebnisse an und lesen Sie nur Sprachen der Form $L_{r,0,t}$ direkt ab.
\\ Hinweis: Wenn Sie frühzeitig $L_{q_3, 2, q_2}$ berechnen, können Sie sich einige Rechenschritte sparen.
\begin{figure}[H]
\begin{center}
\begin{tikzpicture}[scale=0.6,node distance=1.9cm,shorten >=1pt,auto]
\node[state,initial]	(q_1)				 	{$q_1$};
\node[state,accepting]	(q_2)	[right of=q_1]	{$q_2$};
\node[state] 			(q_3)	[right of=q_2]	{$q_3$};
\path[->]	(q_1) 	edge [bend left]	node [above] {$a,b$} 	(q_2)
			(q_2)	edge [bend left]	node [below] {$b$}		(q_1)
			(q_2)	edge 				node {$a$}				(q_3)
			(q_3)	edge [loop above]	node {$a,b$}			(q_3);
\end{tikzpicture}
\end{center}
\end{figure}
\end{frame}

%Pumpin' Lemma
\section{Pumping Lemma}
\subsection{Pumpin' Lemma}
\begin{frame}
\frametitle{Pumping Lemma}
\begin{exampleblock}{Pumping Lemma}
Sei $L$ eine reguläre Sprache. Dann existiert eine Zahl $n \in \mathbb{N}$, sodass für jedes Wort $w \in L$ mit $\left|w \right| > n$ eine Darstellung $$w = uvx$$ existiert, so dass folgende Eigenschaften erfüllt sind:

\begin{enumerate}
\item $v \neq \varepsilon$ 
\item $\left|uv\right| \leq n$ 
\item Für alle $i \in \mathbb{N}_0$ gilt: $uv^ix \in L$
\end{enumerate}
\end{exampleblock}

\begin{center}
\includegraphics[width=0.55\textwidth]{images/Q116}
\end{center}

\end{frame}

\begin{frame}
\frametitle{Pumping Lemma: Übersicht}
\begin{itemize}
\item Jede reguläre Sprache erfüllt das Pumping Lemma. Aber: Nicht jede Sprache, die das Pumping Lemma erfüllt, ist regulär!
\item In der Übung wird üblicherweise die Kontraposition des Pumping-Lemmas verwendet: Man zeigt für eine Sprache, dass das Pumping-Lemma \emph{nicht} erfüllt ist, woraus folgt, dass diese Sprache \emph{nicht} regulär sein kann.

\pause\item $ \neg\left[ \exists n \in \N \,:\, \forall w \in L, |w| > n \,:\, \exists uvx=w \,:\, \ldots \forall i \in \N \,:\, uv^i x \in L\right] $ \\ $ \Leftrightarrow \forall n \in \N \,:\, \exists w \in L, |w| > n \,:\, \forall uvx=w \,:\, \ldots \exists i \in \N \,:\, uv^i x \not\in L $

\pause\begin{itemize}
\item Finden wir für \emph{jedes} $n$ \emph{ein} $w$ mit $\left|w\right| > n$, so dass für \emph{jede} Darstellung $w = uvx$ mit $v \neq \varepsilon$ sowie $\left|uv\right| \leq n$ ein $i \in \mathbb{N}_0$ existiert mit $uv^ix \notin L$, dann ist $L$ nicht regulär.
\end{itemize}
\end{itemize}

\end{frame}

\begin{frame}
\frametitle{Beispiel}
Sei $\Sigma = \{a, b\}$ und $L = \{a^nb^n\,|\,n\geq0\}$. (Also $L = \{\varepsilon,ab, aabb, aaabbb, \ldots\}$)
\begin{enumerate}[<+->]
\item Sei $n \in \N$ beliebig, aber fest.
\item Wähle $w = a^nb^n$.
\item Es ist also $\left|w\right| > n$.
\item Nun ist aber für \emph{jede} Darstellung $w = uvx$ mit $\left|uv\right| \leq n$ und $v \neq \varepsilon$ $v = a^m$ mit $m \geq 1$. Demnach ist $uv^0x = a^lb^n \neq L$, da $l < n$.
\item Daher kann $L$ nicht regulär sein.
\end{enumerate}

\end{frame}

\begin{frame}
\frametitle{Pumping Lemma: Aufgaben}
Welche der folgenden Sprachen sind regulär? Begründen Sie Ihre Antwort.

\begin{enumerate}
\item Die Menge aller Wörter über $\{0, 1\}$, sodass auf jede Null eine Eins folgt
\item $L_1 = \{ww^R | w \in \{a,b\}^*\}$, wobei $w^R$ das 'Spiegelwort' zu $w$ ist (Sprache der Palindrome gerader Länge)
\item $L_2 = \{a^ib^jc^k \, | \, i < j < k\}$.
\end{enumerate}

\end{frame}

\section{Minimierung und Äquivalenzklassenautomat}
\subsection{Minimierung von DEAs}
%Der Automat hier ist vom vorherigen Beispiel kopiert, aber er sollte den Zweck erfüllen.
\begin{frame}
 \frametitle{Minimierung von DEAs}
 \begin{block}{Beispielautomat \(A = (Q, \Sigma, \delta, s, F)\)}
 \begin{figure}[H]
\begin{center}
\begin{tikzpicture}[scale=0.5,node distance=1.9cm,shorten >=1pt,auto]
\node[state,initial]   (q_1)                {$q_1$};
\node[state,accepting] (q_2) [right of=q_1] {$q_2$};
\node[state,accepting] (q_3) [right of=q_2] {$q_3$};
\node[state,accepting] (q_4) [right of=q_3] {$q_4$};

\path[->]	(q_1) 	edge 			node {$a$} 		(q_2)
		(q_2)	edge 			node {$b$}		(q_3)
		(q_3)	edge [loop above]	node {$b$}		()
		(q_4)	edge			node {$b$}		(q_3);
\end{tikzpicture}
\end{center}
\end{figure}
\end{block}
Kann $A$ auf einen Automaten $A' = (Q', \Sigma, \delta', s', F')$ mit $L(A) = L(A')$ und $|Q'| < |Q|$ überführt werden?
\end{frame}
\begin{frame}
\frametitle{Minimierung von DEAs}
$q_4$ ist nicht vom Startzustand aus erreichbar
\begin{block}{Entferne \(q_4\)}
 \begin{figure}[h]
\begin{center}
\begin{tikzpicture}[scale=0.5,node distance=1.9cm,shorten >=1pt,auto]
\node[state,initial]   (q_1)                {$q_1$};
\node[state,accepting] (q_2) [right of=q_1] {$q_2$};
\node[state,accepting] (q_3) [right of=q_2] {$q_3$};

\path[->]	(q_1) 	edge 			node {$a$} 		(q_2)
		(q_2)	edge 			node {$b$}		(q_3)
		(q_3)	edge [loop above]	node {$b$}		();
\end{tikzpicture}
\end{center}
\end{figure}
\end{block}
\pause
Schon minimal?
\end{frame}
\begin{frame}
 \frametitle{Minimierung von DEAs}
Nein!
\begin{block}{Automat \(A'' = (Q'', \Sigma, \delta'', s, F'')\)}
 \begin{figure}[H]
\begin{center}
\begin{tikzpicture}[scale=0.6,node distance=1.9cm,shorten >=1pt,auto]
\node[state,initial]   (q_1)                {$q_1$};
\node[state,accepting] (q_2) [right of=q_1] {$q_2$};

\path[->]	(q_1) 	edge 			node {$a$} 		(q_2)
		(q_2)	edge [loop above]	node {$b$}		();
\end{tikzpicture}
\end{center}
\end{figure}
\end{block}
\begin{block}{}
 Akzeptierte Sprache: \(L(A) = L(A') = L(A'') = L(ab^*) \)
\end{block}
\end{frame}
\begin{frame}
 \frametitle{Methodischer Ansatz}
 \begin{block}{Definition}
  Zustände eines (deterministischen) endlichen Automaten, die vom Anfangszustand aus nicht erreichbar sind, heißen überflüssig.
 \end{block}
\begin{block}{Vorgehen}
 \begin{enumerate}
  \item Tiefensuche durchführen, um überflüssige Zustände zu finden.
  \item Diese entfernen.
  \item Aus restlichen Zuständen den Äquivalenzklassenautomaten bilden.
 \end{enumerate}
\end{block}
\end{frame}

\subsection{Äquivalenzklassenautomat}

\begin{frame}
 \frametitle{Äquivalenz}
 \begin{block}{Aus der Vorlesung}
  \begin{itemize}
   \item Zwei Zustände haben dasselbe Akzeptanzverhalten, wenn es für das Erreichen eines Endzustandes durch Abarbeiten eines Wortes $w$
   unerheblich ist, aus welchem der beiden Zustände wir starten.
  \end{itemize}
 \end{block}
 \begin{block}{Definition (Äquivalenz):}
  Zwei Zustände $p$ und $q$ eines deterministischen endlichen Automaten heißen \emph{äquivalent} ($p \equiv q$),
  wenn für alle Wörter $w\in\Sigma^*$ gilt:
  \[
   \delta(p, w)\in F \Leftrightarrow \delta(q, w)\in F
  \]
  Offensichtlich ist $\equiv$ eine Äquivalenzrelation. Mit $[p]$ bezeichnen wir die Äquivalenzklasse der zu $p$ äquivalenten Zustände.
 \end{block}
\end{frame}
\begin{frame}
 \frametitle{Beispiel}
 \begin{block}{Zurück zu $Q'$}
 \begin{figure}[h]
\begin{center}
\begin{tikzpicture}[scale=0.5,node distance=1.9cm,shorten >=1pt,auto]
\node[state,initial]   (q_1)                {$q_1$};
\node[state,accepting] (q_2) [right of=q_1] {$q_2$};
\node[state,accepting] (q_3) [right of=q_2] {$q_3$};

\path[->]	(q_1) 	edge 			node {$a$} 		(q_2)
		(q_2)	edge 			node {$b$}		(q_3)
		(q_3)	edge [loop above]	node {$b$}		();
\end{tikzpicture}
\end{center}
\end{figure}
Hier sind $q_2$ und $q_3$ äquivalent. Warum?
\end{block}
\end{frame}
\begin{frame}
 \frametitle{Äquivalenzklassenautomat}
 \begin{block}{Definition aus der Vorlesung (Äquivalenzklassenautomat)}
  Zu einem DEA \(A = (Q, \Sigma, \delta, s, F)\) definieren wir den Äquivalenzklassenautomaten 
  \(A^\equiv = (Q^\equiv, \Sigma^\equiv, \delta^\equiv, s^\equiv, F^\equiv)\) durch:
  \begin{itemize}
   \item $Q^\equiv := \menge{[q]}{q\in Q}$
   \item $\Sigma^\equiv := \Sigma$
   \item $\delta^\equiv([q], a) := [\delta(q, a)]$
   \item $s^\equiv := [s]$
   \item $F^\equiv:= \menge{[f]}{f\in F}$
  \end{itemize}
 \end{block}
\end{frame}
\begin{frame}
 \frametitle{Konstruktion der Äquivalenzklassen}
 \begin{block}{}
  \begin{enumerate}
   \item Fasse alle Zustände $q_i \in Q$ in einer Klasse zusammen.
   \item $\varepsilon$ trennt Zustände aus $F$ von denen aus $Q \setminus F$.
   \item Für Worte $w\in \Sigma^*$ mit wachsender Länge und Zustandspaare $p, q$ in einer Klasse: 
    \begin{enumerate}
    \item Falls $[\delta(p, w)] \neq [\delta(q, w)]$, trenne die Zustände $q$ und $p$ ($w$ ist Zeuge).
    \item Brich ab, falls sich für eine Wortlänge keine weiteren Zeugen finden.
    \end{enumerate}
  \end{enumerate}
 \end{block}
 \pause
 \begin{block}{Toller:}
   \begin{enumerate}
    \setcounter{enumi}{2}
    \item Für Zeichen $x \in \Sigma$ und Zustandspaare $p, q$ in einer Klasse:
     \begin{enumerate}
     \item Falls $[\delta(p, x)] \neq [\delta(q, x)]$, trenne die Zustände $q$ und $p$ .
     \item Brich ab, falls in einem Durchlauf kein Zeichen mehr Zustände trennt.
     \end{enumerate}
   \end{enumerate}
  \end{block}
\end{frame}
\frame{
  \frametitle{Automaten-Minimierung: Beispiel}
  \begin{figure}[H]
  \begin{center}
  \begin{tikzpicture}[node distance=2.3cm]
  \tikzstyle{normal}=[draw]


  \node[state,accepting,initial]  (q0)                      {$q_0$};
  \node[state]         (q1)  [above right of=q0] {$q_1$};
  \node[state]         (q2)  [right of=q1]       {$q_2$};
  \node[state]         (q3)  [below right of=q0] {$q_3$};
  \node[state]         (q4)  [right of=q3]       {$q_4$};
  \node[state]         (q5)  [above right of=q4] {$q_5$};
  
  \draw[->,bend left] (q0)  to node[above] {1} (q1);
  \draw[->,loop below] (q0)  to node [below] {0} (q0);

  \draw[->,bend left] (q1)  to node[below] {1} (q0);
  \draw[->] (q1)  to node[above] {0} (q2);

  \draw[->] (q2) to node[above] {0} (q3);
  \draw[->,loop right] (q2) to node[right] {1} (q2);

  \draw[->] (q3)  to node[above] {1} (q0);
  \draw[->,bend left] (q3)  to node[above] {0} (q4);
  
  \draw[->,bend left] (q4)  to node[below] {0} (q3);
  \draw[->] (q4)  to node[right] {1} (q2);

  \draw[->] (q5)  to node[right] {0} (q2);
  \draw[->] (q5)  to node[right] {1} (q4);
  \end{tikzpicture}
  \end{center}
  \end{figure}
}
\begin{frame}
 \frametitle{Automaten-Minimierung: Aufgabe}
Konstruiere zu folgendem DEA den Äquivalenzklassenautomaten:

 \begin{figure}[H]
  \begin{center}
  \begin{tikzpicture}[node distance=2cm]
  \tikzstyle{normal}=[draw]


  \node[state,initial]         (q0)                      {$q_0$};
  \node[state]         (q1)  [right of=q0] {$q_1$};
  \node[state]         (q2)  [below of=q0]       {$q_2$};
  \node[state]         (q3)  [right of=q2] {$q_3$};
  \node[state]         (q4)  [right of=q3]       {$q_4$};
  \node[state,accepting]         (q5)  [right of=q4] {$q_5$};

  \draw[->] (q0)  to node[above] {1} (q1);
  \draw[->] (q0)  to node [left] {0} (q2);

  \draw[->,loop right] (q1)  to node[right] {0} (q0);
  \draw[->] (q1)  to node[right] {1} (q3);

  \draw[->] (q2) to node[below] {1} (q3);
  \draw[->,loop left] (q2) to node[below] {0} (q2);

  \draw[->] (q3)  to node[above] {1} (q4);
  \draw[->] (q3)  to node[above] {0} (q0);
  
  \draw[->,loop above] (q4)  to node[right] {0} (q4);
  \draw[->] (q4)  to node[above] {1} (q5);

  \draw[->,loop above] (q5)  to node[right] {0, 1} (q2);
  \end{tikzpicture}
  \end{center}
  \end{figure}
\end{frame}

\section{Wiederholung}
\subsection{DEAs}

\begin{frame}
\frametitle{Aufgabe zu DEAs}
Gegeben sei ein DEA (als Graph mit beschrifteten Knoten und Kanten) ohne überflüssige Zustände.
Schreibe einen Algorithmus, der entscheidet, ob die zugehörige Sprache unendlich viele Wörter hat. Hinweis: Siehe Beweis zur Korrektheit des Pumping-Lemmas.
\end{frame}



\subsection{Reguläre Sprachen}

\begin{frame}
\frametitle{Aufgabe: Regularität}

\begin{block}{Definition}
Ein Wort $w$ ist \emph{Präfix} eines Wortes
$w'$, falls ein Wort $v$ mit $wv = w'$ existiert. Gilt zusätzlich $w\neq w'$, so ist $w$ ein \emph{echtes Präfix} von $w'$.
\end{block}

\pause

\begin{block}{Aufgabe}
Sei $L$ eine reguläre Sprache. Zeige, dass dann auch die folgenden beiden Sprachen regulär sind:
\begin{enumerate}
\item NOPREFIX$(L):=\menge{w\in L}{\mbox{kein $w'\in L$ ist echtes Präfix
    von $w$}}.$
\item NOEXTEND$(L):=$ \\ \raggedleft{ $ \menge{w\in L}{\mbox{$w$ ist kein echtes Präfix eines Wortes $w'\in L$}}.$ }
\end{enumerate}
\end{block}

\pause

\begin{block}{Zur Veranschaulichung}
$ L = \left\lbrace \text{ Katze, Kater, Katzenpfote, Katzenfutter } \right\rbrace $

Was sind hier NOPREFIX$(L)$ und NOEXTEND$(L)$?
\end{block}

\end{frame}
\begin{frame}
 \frametitle{Aufgabe: Regularität}
 Zeige, dass die folgende Sprache nicht regulär ist:
 $$L = \menge{0^m 1^n}{m \neq n}$$
\end{frame}

\section{Schluss}
\subsection{Schluss}

\begin{frame}
 \frametitle{Bis zum nächsten Mal!}
 \begin{center} \includegraphics[scale=0.5]{images/xkcd_336.png} \end{center}
 \begin{quote}\scriptsize{You should start giving out 'E's so I can spell FACADE or DEFACED.}\end{quote}
\end{frame}

\include{includes/common_end}
